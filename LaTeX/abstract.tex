\begin{abstract}
Non-photorealistic rendering (NPR) is a combination of computer graphics and artistic techniques used in 
a variety of applications. These techniques can be used to emphasize visual information from an initial photorealistic rendering, or to present the rendering in a different artistic style. Modifications to the initial model can aim to emphasize details, create a particular mood, or both.

% Our goal is to create a system and algorithm to render 3D models in a style similar to 2D animation 
% using OpenGL and shaders. Transforming the rendered model will be achieved through cel shading, 
% lighting effects, modifying textures, and adding contour lines to the image. We will also examine the 
% effect of applying these transformations to the model in both the vertex and fragment shaders.

In this project, we investigate techniques used for NPR, and apply them to our own renderings of 3D 
models. In particular, we attempt to reproduce the style of 2D animation with 3D models by applying 
cel shading, lighting effects, sampling textures, contours, and suggestive contours. For our 
implementation, we use the OpenGL pipeline to apply a rendering process with several stages, in order
to produce our final cartoon renderings. We explore the effects of using our system on models with different properties, and show that the effectiveness of simple algorithms is highly dependant on the shape and composition of the model.

% TODO: Should summarize our results and conclusions after these sections are written.
\end{abstract}
\begin{abstract}
Non-photorealistic rendering is a combination of computer graphics and artistic techniques used in a variety of applications to emphasize information from a photorealistic rendering or present it in a different style. This can be done either through artistic techniques that create a particular mood, or by emphasizing details of the image/model, or both.

Our goal is to create a system and algorithm to render 3D models in a style similar to 2D animation using OpenGL and shaders. Transforming the rendered model will be achieved through cel shading, lighting effects, modifying textures, and adding contour lines to the image. We will also examine the effect of applying these transformations to the model in both the vertex and fragment shaders.
\end{abstract}